\chapter{Title of the chapter}
\section{Introduction}
\blindtext
\section{Nomenclature examples}
courant continu (DC\nomenclature[a-dictCur]{DC}{Courant continu (Direct current)}), courant alternatif (AC\nomenclature[a-AltCur]{AC}{Courant alternatif (Alternating current)}), le système photovoltaïque (PV) \nomenclature[a-phot]{PV}{Photovoltaïque}
\begin{equation}\label{Iph}
I_{ph}=\frac{G}{G_r}\left(I_{sc}+\alpha_{sc}(T_{cell}-T_{ref})\right)
\end{equation}
%%
\nomenclature[s-Ecl]{$G$}{Eclairement sur la surface de la cellule}
\nomenclature[s-Eclref]{$G_r$}{Eclairement de référence (1000 \si{\watt/\meter\square})}
\nomenclature[s-ISC]{$I_{sc}$}{Courant du court-circuit (short-circuit current)}
\nomenclature[g-CoefTisc]{$\alpha_{sc}$}{Coefficient de température de $I_{sc}$}
\nomenclature[s-TempCell]{$T_{cell}$}{Température sur la surface de la cellule}
\nomenclature[s-TempCellRef]{$T_{ref}$}{Température de référence (25 \si{\degreeCelsius})}
%%
\\
\begin{equation}
	E=\frac{1}{2}mv^2
	\nomenclature[s-ener]{$E$}{ Energy}
	\nomenclature[s-mas]{$m$}{ Mass}
	\nomenclature[s-velo]{$v$}{ Velocity}
\end{equation}
Use \verb|	\nomenclature[s-ener]{$E$}{ Energy}| to add a  symbols in the Nomenclature List

The \verb|s| for the type and \verb|ener| for alphabetical order
\verb|{$E$}| for the symbol and \verb*|{Energy}| the signification of the symbols

\section{Figures and Tables examples}
Dans la \fig{figLabel}, et dans le \tab{tabLabel}.
\begin{figure}[H]
	\centering
	\includegraphics[width=0.55\linewidth,scale=1.0]{5-Chapters/Chapter1/drawing.eps}
	\caption[Short title of the figure]{Long title of the figure\label{figLabel}}
\end{figure}

\begin{table}[H]
	\centering
	\caption{Les paramètres du module photovoltaïque 
		\label{tabLabel}}
	\begin{tabular}{@{\extracolsep{\fill}}lll@{}}
		\toprule
		Paramètres & Valeurs & Unités\\
		\midrule			
		$P_{mp}$& 100& \si{\watt}\\
		\midrule
		$V_{oc}$& 13.3& \si{\volt}\\
		\midrule
		$I_{sc}$& 9.8& \si{\ampere}\\							
		\bottomrule
	\end{tabular}
\end{table}
\section{Conclusion}
\blindtext